\documentclass{article}
\usepackage[danish]{babel}
\usepackage{amsmath}
\usepackage{float}
\usepackage{caption}
\usepackage{placeins}
\usepackage{graphicx}

\usepackage[utf8]{inputenc}

\title{Afleveringsopgave1}
\author{Nicolas Kaveh Vahman s164404\\
Michelle Westphal Sørensen s164446\\
Marie-Louise Wolfsberg Schmidt s164417}\\
\date{30. September 2016}

\begin{document}

\maketitle

\begin{abstract}
Dette dokument omhandler boblesortering. Der beskrives algoritmen
og præsenteres en kompleksitetsanalyse.

\end{abstract}
\section{Introduktion}


Boblesortering (eng. bubble sort) er en populær sorteringsalgoritme og er en af
de simpleste algoritmer at forstå og implementere. Dog er den ikke en særlig
effektiv sorteringsalgoritme\footnote{Mere om dette i “Algoritmer og Datastrukturer 1”}
hverken for store eller små lister, og den anvendes
sjældent i praksis. Boblesortering sorterer, som navnet antyder, elementerne i
en liste ved at boble hvert element gennem listen til sin rette plads i listen.

\subsection{Pseudokode}

\begin{verbatim}
    
Wikipedia [2] giver følgende pseudokode for boblesortering.

procedure bubbleSort( A : list of sortable items ) defined as:
    do
      swapped := false
      for each i in 0 to length(A) - 2 inclusive do:
        if A[i] > A[i+1] then
         swap( A[i], A[i+1] )
         swapped := true
       end if
     end for
   while swapped
end procedure
\end{verbatim}







\section{Analyse af boblesortering}

Antallet af sammenligninger, som boblesortering udfører på en tabel af længde n,
er i værste fald

\[\sum\limits_{i = 1}^{n - 1} {i = 1 + 2 + 3 + ... + n - 1 = n\frac{{n(n - 1)}}{2}.} \]

I bedste fald er antallet. Se tabel 1



\begin{figure}[H]
    \centering
     \label{fig:my_label}
\includegraphics[scale=1.0]{Udklipboble.JPG}
\caption{Illustration af boblesortering}
\end{figure}


\begin{figure}[H]
\centering
\begin{tabular}{|r|l|}
\hline
7C0 & hexadecimal\\
\hline
3700 & octal\\
\hline
11111000000 & binær\\
\hline
1984 & decimal\\
\hline

\end{tabular}    
\end{figure}



\section{Litteratur}


\begin{thebibliography}{1}

\bibitem{donald}
Donald Knuth,
The Art of Computer Programming
Volume 3. Addison - Wesly


\bibitem{wiki}


\end{thebibliography}

\end{document}
